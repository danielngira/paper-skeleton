\section{Conclusion}\label{sec:conclusion}

Our analysis strongly suggests that the internet disruption observed in Tanzania from October 29 to November 1, 2025, was a deliberate, state-sponsored shutdown rather than a technical failure or deep-sea cable outage. By leveraging data from Georgia Tech’s IODA project, we established that the outage was geographically isolated to Tanzania. Unlike the 2024 disruption in Kenya (where simultaneous outages in neighboring countries corroborated a subsea cable failure) our cross-reference of network traffic in Kenya, Mozambique, and Somalia during this period showed no comparable anomalies. Furthermore, OONI connectivity data revealed a targeted pattern of censorship: specific platforms (Twitter, WhatsApp, Signal) experienced distinct blocks and "throttling" anomalies that coincided precisely with the election timeline and subsequent political unrest. This granular targeting of communication tools, particularly those offering encryption (Signal, WhatsApp), further indicates a strategic intent to stifle political discourse and coordinate surveillance rather than a generalized infrastructure failure.

Based on these findings, several remedial and preventive actions are necessary to protect digital rights in the regions. Organizations such as the Zaina Foundation and international watchdogs must continue to preserve forensic network data (like the IODA and OONI traces collected in this report). This evidence is critical for filing legal challenges against the Tanzania Communications Regulatory Authority (TCRA) and for supporting potential sanctions by the African Commission on Human and Peoples' Rights (ACHPR). Service providers like Signal and Meta (WhatsApp) are uniquely equipped to implement censorship-resistant routing (such as domain fronting) for users in high-risk regions. Additionally, these platforms should publish real-time transparency reports confirming traffic drops to publicly refute government denials of interference.

