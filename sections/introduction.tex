\section{Introduction}\label{sec:intro}

In the last decade, cases of internet shutdowns have been used as a tool for digital repression in Sub-Saharan Africa. Governments have increasingly restricted connectivity during moments of political tension, particularly during national elections. The frequency of shutdowns rose from 12 incidents in 2017 to 25 in 2019 (“Internet Blockings and Government Revenue: Case Study Tanzania”). Tanzania has popped up as a major player in this trend. On October 29th, 2025, the country experienced a nationwide disruption that affected mobile data and major social media platforms. This event was confirmed by real-time network telemetry from NetBlocks, which noted a steep drop in traffic on the morning of the election (“Business and Human Rights Center”). These disruptions happened during protests over restrictive electoral conditions and the suppression of opposition candidates, which reinforced concerns that connectivity controls are being used as a tool of political containment instead of ensuring public safety.

These actions signal a larger pattern in Tanzania’s digital governance landscape. In recent years, authorities have introduced increasingly stringent forms of online regulation. The Electronic and Postal Communications (Online Content) Regulations of 2017 have criminalized content that is deemed “indecent,” “obscene,” or “disruptive to public order”. This categorization grants authorities a wide scope for censorship and has already resulted in arrests for anti-government expression (“TechCabal”). These regulations require all online content creators, including bloggers and YouTubers, to register and pay licensing fees, and have been used to justify actions such as the suspension of Mwananchi Communications’ digital division in 2024 for posting an animated video about political abductions (“TechCabal”). Government framing of these policies as “child protection” measures has been widely criticized as a pretext for broader censorship. As the Global Network Initiative notes, even mainstream platforms such as WhatsApp, X, and Instagram were rendered inaccessible during the election period (“GNI Statement on the Nationwide Internet Shutdown in Tanzania”).

The stakes of such shutdowns extend far beyond freedom of expression. Tanzania’s 2016 ICT Policy places the internet as essential infrastructure for economic development, financial inclusion, and employment, with a national goal of achieving 80% broadband access by 2025 (“Internet Blockings and Government Revenue: Case Study Tanzania”). The country’s informal sector, which represents 76% of the national workforce, relies heavily on mobile phones, mobile money, online marketplaces, and app-based services (“Internet Blockings and Government Revenue: Case Study Tanzania”). Previous analyses show that even short disruptions hinder business operations, interrupt digital payments, reduce advertising reach, and constrain the functioning of electronic fiscal devices used for tax collection.

Given Tanzania’s position as the home of Dar-es-Salaam, one of East Africa’s largest financial hubs, the effects of a nationwide shutdown spill across borders. Internet disruptions interfere with regional commerce, international banking, and trade logistics. Meanwhile, concerns about state surveillance, intercepted communications, and politically motivated arrests further contribute to an environment in which digital repression shapes civic behavior (“Business and Human Rights Center”).
In this context, our study conducts an empirical analysis of the October 2025 Tanzanian internet shutdown, combining OONI web-connectivity tests, IODA outage measurements, and cross-national comparison to assess the intentionality, scope, and impact of state-imposed connectivity restrictions. Our analysis contributes to ongoing debates about digital authoritarianism, electoral integrity, and the economic and political consequences of network disruptions.