\section{Related Work}\label{sec:related}

Existing work on internet shutdowns in Africa spans several domains, including political repression, digital censorship, human rights, and economic impact. Studies have noted the rise in internet connectivity restrictions by African governments, especially during election cycles or episodes of civil unrest. For example, comparative analyses of shutdowns in Uganda, Ethiopia, Zimbabwe, and Togo highlight how governments strategically time disruptions to suppress mobilization and information-sharing (“Internet Blockings and Government Revenue: Case Study Tanzania”). Similarly, work by civil society groups such as NetBlocks and the Business & Human Rights Resource Centre provides real-time documentation of shutdowns, including traffic anomalies and platform-specific blocking patterns, which has been instrumental in verifying the October 2025 Tanzanian disruption (“Business and Human Rights Center”).

Research on Tanzania specifically shows a pattern of regulations aimed at controlling digital spaces. Reports from TechCabal highlight the shutdown of more than 80,000 websites in 2025 and earlier cases of punitive actions against journalists and content creators under the Electronic and Postal Communications (Online Content) Regulations of 2017, which necessitate the licensing of online publication and criminalize broad categories of speech (“TechCabal”). These findings align with broader analyses of digital repression across the continent, where policies ostensibly intended to regulate harmful or illegal content are leveraged to silence dissent.

From a human rights perspective, the Global Network Initiative documents instances in which entire platforms like WhatsApp, Instagram, and X were made inaccessible in Tanzania during politically sensitive periods (“GNI Statement on the Nationwide Internet Shutdown in Tanzania”). Research on online participation under censorship further suggests that such blocks significantly alter user behavior, suppress political speech, and disrupt civil society networks.
Economic analyses of internet shutdowns in Africa remain limited, but the existing work emphasizes the substantial financial consequences of connectivity restrictions. The SSRN study “Internet Blockings and Government Revenue: Case Study Tanzania” employs an interrupted time-series model to demonstrate that while Tanzania’s 2015 election (no shutdown) saw increases in tax and total revenue, the 2020 shutdown coincided with reductions in revenue, disruption of VAT collection, and impaired functionality of electronic fiscal devices. Related work on the “internet economy” (Barua et al., Blinder, Kogut) highlights how digital connectivity drives productivity, financial inclusion, and informal sector growth. This suggests that shutdowns disrupt economic fundamentals rather than isolated digital platforms.

The technical analyses of outages use network measurement frameworks such as OONI and IODA to differentiate between deliberate government interference and alternative explanations like submarine cable failures. A recent Georgia Tech IODA report on the 2024 Kenya outage provides a methodological template for this approach, comparing cross-country signals to rule out cable damage. Our study applies a similar technique, referencing IODA’s cross-national dashboards to compare Tanzania’s traffic patterns with those of Kenya, Mozambique, and Somalia, none of which experienced comparable disruptions during the same window (“IODA: The Kenya June 25, 2024 Internet Disruption”).

By bringing together economic, political, and network-measurement perspectives, the literature shows that shutdowns are rarely technical accidents. Instead, they are complex political interventions with measurable social, civic, and economic costs. Our study builds on these works by combining platform-specific blocking evidence from OONI, country-level reachability data from IODA, and regional comparative analysis to evaluate the intentionality and broader implications of Tanzania’s October 2025 shutdown. 
